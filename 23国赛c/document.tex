\documentclass[withoutpreface,bwprint]{cumcmthesis} %去掉封面与编号页
\usepackage[framemethod=TikZ]{mdframed}
\usepackage{url}   % 网页链接
\usepackage{subcaption} % 子标题、
\usepackage{graphicx}
\title{“信息引航者”——基于MGCA的新闻推荐系统}
\usepackage{listings}
\lstset{language=Matlab}
\usepackage{pythonhighlight}
\usepackage{setspace}
\begin{document}
	
	\maketitle\thispagestyle{empty}
	\begin{abstract}
		信息技术不断发展,互联网上存在着大量信息,通过小红书、今日头条、bilibili等互联网平台获取信息是主要的信息获取途径之一。随着互联网上信息的越来越多,想要在海量信息中获取我们感兴趣或有需求的信息愈加困难。\par
		推荐系统的出现正是为了解决这种“信息过载”的问题。推荐系统已在互联网中得到了广泛的应用,并给应用它的企业带来了丰厚的利润。推荐系统给亚马逊带来了35\%的销售收入,给Netflix带来了高达75\%的消费,并且Youtube主页上60\%的浏览来自推荐服务。有关推荐系统的研究具有十分深远的意义与巨大的实用价值。\par
		
	\end{abstract}
	\setcounter{page}{1}
	\tableofcontents
	\newpage
	\section{引言}
	\subsection{研究背景及意义}
	\subsubsection{ “信息过载”问题的解决迫在眉睫}    
	\subsubsection{ 互联网与推荐系统的融合带来巨大的经济价值}
	\subsection{相关研究现状}
	\subsubsection{ 推荐系统在工业界的应用现状}
	\subsubsection{ 推荐系统在学术界的研究现状}
	\subsection{项目架构总览}
	\subsubsection{ 项目整体框架}
	\subsubsection{ 项目优势与创新性}
	\newpage
	\section{模型技术路线及实现方案}
	\subsection{方法介绍}
	\subsubsection{ 图神经网络}
	\subsubsection{ 注意力机制}
	\subsubsection{ Transformer}
	\subsubsection{ Fastformer}
	\subsection{模型架构}
	\subsubsection{ 总体架构}
	\subsubsection{ 候选新闻编码}
	\subsubsection{ 单词粒度感知候选新闻}
	\subsubsection{ 新闻粒度感知候选新闻}
	\subsubsection{ 实体粒度感知候选新闻}
	\subsection{模型评估}
	\subsubsection{ 与之前模型的性能对比}
	\subsubsection{ 消融实验}
	\newpage
	\section{软件工程与开发架构方案}
	\subsection{开发工具与框架}
	\subsection{新闻推荐系统软件落地示例}
	\subsubsection{ 软件周期模型}
	\subsubsection{ 软件开发模型}
	\subsubsection{ 数据库设计}
	\subsubsection{ UML设计}
	\subsubsection{ 软件功能演示}
	\subsubsection{ 兼容性测试与稳定性测试}
	$\bigstar$可执行测试:分别部署到不同性能的电脑上运行测试,均成功运行,网页反馈
	均在35ms到45ms左右,不存在延迟和卡顿现象,该系统对电脑的配置要求较低。\par
	$\bigstar$功能测试:所有功能均可正常运行使用,可视化动态大屏显示正常,首页展示
	效果正常,查询功能正常,链接页面跳转正常。\par
	$\bigstar$兼容测试:在 Windows、Linux、Mac等操作系统可以正常访问。谷歌浏览器、
	火狐浏览器、QQ 浏览器、360 浏览器等运行均正常。Windows操作系统的python版
	本为3.6以上、tensorflow2.14.0 版本环境可以正常运行平台。\par
	$\bigstar$安全测试:本项目采取的是 https 协议,安全性和可靠性比较高,jupyter 环境
	提供了三种安全级服务配置,可以按照实际的需求提升安全等级。\par
	\newpage
	\section{商业模式构建与经营管理}
	\subsection{市场竞争分析}
	\subsection{投资回报分析}
	\subsection{盈利方式}
	\subsubsection{ 提供整套低耦合工业界解决方案}
	\subsubsection{ 提供整套成熟软件系统}
	\subsection{财务管理}
	\subsection{营销战略}
	
\end{document}
